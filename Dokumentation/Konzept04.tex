\documentclass[11pt,a4paper]{article} 
\usepackage[utf8]{inputenc}
\usepackage[ngerman]{babel}
\usepackage{amsmath}
\usepackage{amssymb}
\usepackage{geometry}
\geometry{
  left=3cm,
  right=3cm,
  top=2cm,
  bottom=3cm
}

\title{Ausbeutung von Tierpopulationen \\ \textbf{Konzept} }
\author{Jannik Simon Münz, Ruedi Lüthi}

\begin{document}
	\bibliographystyle{plain}
	
	\maketitle
	
	Da sich die Blauwale mit konstanter Fangintensität prächtig entwickeln, wollen wir unser Model um eine weitere Spezies erweitern. Vorerst lassen wir beide unabhängig voneinander anwachsen:
	\begin{flalign*}
		\textrm{Spezies A: } \qquad &
		a_{n+1} = a_{n} + r_{a}~a_{n}  \\
		\textrm{Spezies B: } \qquad &
		b_{n+1} = b_{n} + r_{b}~b_{n}
	\end{flalign*}
	
	Um dem Wachstum eine Grenze zu setzen, nutzen wir wie in der logistischen Wachstumsgleichung beschrieben, eine Umweltkapazität \( K \). Kombiniert mit den Summen der beiden Populationen ergeben sich dann folgende Gleichungen:
	\begin{flalign*}
		\textrm{Spezies A: } \qquad &
		a_{n+1} = a_{n} + r_{a} \left( 1 - \frac{a_{n} + b_{n}}{K} \right) a_{n} \\
		\textrm{Spezies B: } \qquad &
		b_{n+1} = b_{n} + r_{b} \left( 1 - \frac{a_{n} + b_{n}}{K} \right) b_{n}
	\end{flalign*}
	
	Nun sollen unsere zwei Arten auch miteinander interagieren können. Dazu führen wir drei weitere Variablen ein: \( \alpha \) definiert die Wahrscheinlichkeit einer Begegnung von A mit B in unserem Biotop. Sie ist alleine von der Gestalt des Biotops abhängig und für beide Arten gleich. Zusätzlich beschreiben die Variablen \( l_{a} \) und \( l_{b} \) den Effekt der Begegnung für die jeweilige Spezies. Damit erhalten wir:
	 \begin{flalign*}
		\textrm{Spezies A: } \qquad &
		a_{n+1} = a_{n} + r_{a} \left( 1 - \frac{a_{n} + b_{n}}{K} \right) a_{n} + l_{a}~\alpha~x_{a}~x_{b} \\
		\textrm{Spezies B: } \qquad &
		b_{n+1} = b_{n} + r_{b} \left( 1 - \frac{a_{n} + b_{n}}{K} \right) b_{n} + l_{b}~\alpha~x_{a}~x_{b}
	\end{flalign*}	
	Dies führt uns zu den allgemein bekannten Lotka-Volterra-Gleichungen\cite{LotkaVolterra} und zu einem Räuber-Beute-System.
	
	Um einen zusätzlichen Blickwinkel auf das Populationsverhalten dieses Räuber-Beute Systems zu erhalten, möchten wir zusätzlich einen Ansatz verfolgen, welcher einzelne Individuen oder Zellen von kleineren Gruppen simuliert. Als Grundlage dafür soll ein zellulärer Automat\cite{ZellulaererAutomat} dienen. Beiden Modelle möchten wir vergleichen und analysieren in welchen Bereichen diese miteinander korrelieren.
	
	\bibliography{literatur}
	
	\noindent\textit{Selbstverständlich werden diese Wikipedia Quellen noch durch seriöse wissenschaftliche Werke ersetzt.}

\end{document}	