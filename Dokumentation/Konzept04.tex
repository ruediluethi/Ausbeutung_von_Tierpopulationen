\documentclass[11pt,a4paper]{article} 
\usepackage[utf8]{inputenc}
\usepackage[ngerman]{babel}
\usepackage{amsmath}
\usepackage{amssymb}
\usepackage{geometry}

\title{Ausbeutung von Tierpopulationen \\ \textbf{Konzept} }
\author{Jannik Simon Münz, Ruedi Lüthi}

\begin{document}

	\maketitle
	
	\begin{itemize}
		\item Verhalten von mehreren Wachstumspopulationen, welche von einander abhängen. Im wesentlichen sind dabei die Parameter \(r\) und \(K\) Funktionen in Abhängigkeit der anderen Populationsgrösse. 
		
	Räuber, Beute? Lotka-Volterra
		
		\item Zusammenfassen von mehreren Populationen in Zellen und definieren von verschiedenen Schnittstellen. So können beispielsweise Populationen von einer Zelle zu einer anderen Wandern.
		
		\item Das Model mit realen Daten prüfen. Dazu müssen passende Messdaten rechechiert werden. Beispielsweise
			\begin{itemize}
				\item Fischbestände in den Weltmeeren, mögliche Quellen: WWF, Greenpeace
				\item Vogelbestände (wie wird dabei die jährliche Wanderung simuliert? nützt dabei evtl. das Zellenmodel?), mögliche Quellen: Tierschutzverbände, Vogelwarte Sempach (CH)
				\item Zellenwachstum in Bioreaktoren, mögliche Quellen: Studiengang Biotechnologie (Hochschule zhaw Wädenswil CH)
			\end{itemize}
			
		\item Wie würde ein Model ausschauen wo jeweils einzelne Individuen simuliert werden? Inwiefern würde ein solches Model mit dem logistischen Wachstum korrelieren?
		
	\end{itemize}

\end{document}	