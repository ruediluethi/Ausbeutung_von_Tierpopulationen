\documentclass[a4paper,twoside]{article}

\usepackage[utf8]{inputenc}
\usepackage[ngerman]{babel}

\usepackage{epsfig}
\usepackage{subfigure}
\usepackage{calc}
\usepackage{amssymb}
\usepackage{amstext}
\usepackage{amsmath}
\usepackage{amsthm}
\usepackage{multicol}
\usepackage{pslatex}
\usepackage{apalike}
\usepackage{enumitem}
\usepackage{MOSI}     % Please add other packages that you may need BEFORE the MOSI.sty package.

\newcommand{\team}{Jannik Simon Münz, Ruedi Lüthi}
\newcommand{\theme}{Ausbeutung von Tierpopulationen}
\setcounter{page}{1}

\subfigtopskip=0pt
\subfigcapskip=0pt
\subfigbottomskip=0pt

\begin{document}

	\title{Ausbeutung von Tierpopulationen\subtitle{Wachstumsgleichungen im Vergleich mit einem zellulären Automaten} }
	
	\author{\authorname{Jannik Simon Münz und Ruedi Lüthi}}
	
	\keywords{bla bla}

	\abstract{bla bla bla}	
	
	\onecolumn \maketitle \normalsize \vfill

	\section{\uppercase{Die Wachstumsgleichung}}\label{sec:Wachstumsgleichung}

	\noindent Nachfolgend wird das Wachstum einer Blauwalpopulation als Funktion abhängig von der Zeit \(y(t)\) beschrieben. Ohne äußere Einflüsse wächst die Population der Blauwale um einen konstanten Faktor \(r\) an, wobei das Maximum der Population durch die Umweltkapazität \(K\) begrenzt ist. Von Interesse ist hierbei die Veränderung der Population im Zeitintervall \(\Delta t\). Dies führt zu der folgenden Differentialgleichung:
	\begin{align*}
		y(t+\Delta t) &= y(t) + r\cdot\left(1-\frac{y(t)}{K}\right) \cdot y(t) \cdot \Delta t \\
		%&= \left(1 + r\cdot\left(1-\frac{y(t)}{K}\right) \cdot \Delta t \right) \cdot y(t)
		\frac{y(t+\Delta t) - y(t)}{\Delta t} &= r \cdot\left(1-\frac{y(t)}{K}\right) \cdot y(t) \\
		\dot{y} &= r \cdot\left(1-\frac{y}{K}\right) \cdot y
	\end{align*}

	Für die Population der Blauwale in der Antarktis \cite{Skript} wurden folgende Werte für die beschriebenen Parameter geschätzt: \(r = 0.06, K = 150'000\). Daraus ergibt sich folgendes Bild (siehe Abbildung \ref{fig:wachstum_ohne_einfluesse}).
	
	\begin{figure}[!h]
  		\centering
 		\includegraphics[width=5.5cm]{Diagramme/wachstum_ohne_einfluesse.png}
  		\caption{Population der Blauwale in der Antarktis.}
  		\label{fig:wachstum_ohne_einfluesse}
	\end{figure}

	Dabei bleibt die Population bei \(y = K\) im stabilen Gleichgewicht.

	Im nächsten Schritt wird die Gleichung durch einen äußeren Einfluss, die Fangintensität \(E\) erweitert:
	\begin{align*}
		%y(t+\Delta t) &= y(t) + r \cdot \left(1-\frac{y(t)}{K}\right) \cdot y(t) \cdot \Delta t - E \cdot y(t) \cdot \Delta t \\
		%&= \left(1+r \cdot \left(1-\frac{y(t)}{K}\right)\cdot \Delta t - E \cdot \Delta t \right) \cdot y(t)
		\dot{y} &= r \cdot\left(1-\frac{y}{K}\right) \cdot y - E \cdot y
	\end{align*}	
	
	Anschließend wird die neue Gleichung in die Form der Anfangsgleichung gebracht:
		
	\begin{align*}
		%&= \left(1+r \cdot \left(\underbrace{1 - \frac{E}{r}}_{\textrm{ausklammern}} - \frac{y(t)}{K} \right) \cdot \Delta t \right) \cdot y(t) \\		
		%&= \left(1+\underbrace{r \cdot \left( 1 - \frac{E}{r} \right)}_{\tilde{r}} \cdot \left(1 - \frac{y(t)}{\underbrace{K \cdot \left( 1 - \frac{E}{r} \right)}_{\tilde{K}}} \right) \cdot \Delta t \right) \cdot y(t) \\		
		\dot{y} &= \left( r \cdot\left(1 - \frac{y}{K}\right) - E\right) \cdot y \\
		&= r \cdot \left( 1 - \frac{E}{r} - \frac{y}{K} \right) \cdot y \\
		&= \underbrace{r \cdot \left( 1 - \frac{E}{r} \right)}_{\tilde{r}} \cdot \left( 1 - \frac{y}{\underbrace{K \cdot \left(1 - \frac{E}{r}\right)}_{\tilde{K}}} \right) \cdot y
	\end{align*}
	
	Dabei beschreibt das \(\tilde{r}\) wiederum das proportionale Wachstum und \(\tilde{K}\) das Maximum der Populationsgröße. Den höchsten Ertrag erhält man, wenn sich die Population im stabilen Gleichgewichtspunkt \(y = \tilde{K} = K \cdot \left(1 - \frac{E}{r}\right) \) befindet. Somit wird der maximale Betrag durch \(E \cdot K \cdot \left(1 - \frac{E}{r}\right)\) beschrieben. Die Parameter \(r\) und \(K\) sind bekannt und um nun das maximale \(E\) zu finden, muss das Maximum der Funktion \(f(E) = E \cdot K \cdot \left(1 - E \cdot \frac{1}{r}\right)\) gesucht werden:
	\begin{align*}
		&f'(E) = K - 2 \cdot E \cdot \frac{K}{r} = K \left(1 - E \cdot \frac{2}{r} \right) \stackrel{!}{=} 0 \\
		&\Rightarrow 1 - E \cdot \frac{2}{r} = 0 \Rightarrow E = \frac{r}{2}
	\end{align*}
	Der maximale Ertrag ist somit gegeben durch:
	\begin{align*}
		\frac{r}{2} \cdot K \cdot \left(1 - \frac{r}{2}\cdot\frac{1}{r}\right) = \frac{r \cdot K}{4}
	\end{align*}
	
	\newpage	
	
	Die nachfolgenden Abbildungen \ref{fig:zuwachs_etrag_auf_zeit1} und \ref{fig:zuwachs_etrag_auf_zeit2} zeigen den Zuwachs und den Ertrag über die Zeit für zwei unterschiedliche \(E\)'s.
	\begin{figure}[!h]
  		\centering
 		\includegraphics[width=5.5cm]{Diagramme/zuwachs_etrag_auf_zeit1.png}
  		\caption{Zuwachs und Ertrag für \(E = \frac{r}{2}\).}
  		\label{fig:zuwachs_etrag_auf_zeit1}
	\end{figure}
	\begin{figure}[!h]
  		\centering
 		\includegraphics[width=5.5cm]{Diagramme/zuwachs_etrag_auf_zeit2.png}
  		\caption{Zuwachs und Ertrag für \(E = \frac{r}{3}\).}
  		\label{fig:zuwachs_etrag_auf_zeit2}
	\end{figure}

	Gut zu erkennen ist in Abbildung \ref{fig:zuwachs_etrag_auf_zeit1}, dass die Differenz von Ertrag und Zuwachs auf Null fällt sobald das Populationsmaximum erreicht ist. Dies geschieht in Abbildung \ref{fig:zuwachs_etrag_auf_zeit2} nicht. Es werden also nicht alle Wale gefangen und das Ertragsmaximum wird nicht erreicht.

	\newpage
	In Abbildung \ref{fig:zuwachs_ertrag_zu_population} ist dies ebenfalls zu erkennen, wenn der Zuwachs und Ertrag als Funktion der Populationsgröße aufzeichnet wird:
	\begin{figure}[!h]
  		\centering
 		\includegraphics[width=5.5cm]{Diagramme/zuwachs_ertrag_zu_population.png}
  		\caption{Zuwachs (ohne äußere Einflüsse) und Ertrag als Funktion der Populationsgröße.}
  		\label{fig:zuwachs_ertrag_zu_population}
	\end{figure}

	Für \(E=\frac{r}{2}\) ist der Schnittpunkt zwischen Zuwachs und Ertrag genau im Maximum der Zuwachsfunktion. Für \(E=\frac{r}{3}\) hingegen leicht daneben, somit wird hierfür das Maximum nicht erreicht.
	
	\newpage	
	
	\section{\uppercase{Räuber und Beute}}\label{sec:Raeuber_Beute}
	
	Zusätzlich zu den Walen wird eine weitere Spezies betrachtet, welche in Wechselwirkung mit den Walen steht, beispielsweise Plankton. Zur Vereinfachung werden im nachfolgenden Text die Population der Wale als Räuber \(w(t)\) und das Plankton als Beute \(v(t)\) bezeichnet. Die Population beider Arten verhält sich wie in Abschnitt \ref{sec:Wachstumsgleichung} beschrieben, wobei jedoch das Populationsmaximum jeweils von der Summe beider Arten abhängt:
	\begin{align*}
		\textrm{Räuber (Wale): } &\quad \dot{w} = -r_w \cdot \left(1 - \frac{w+v}{K} \right) \cdot w \\
		\textrm{Beute (Plankton): } &\quad \dot{v} = +r_v \cdot \left(1 - \frac{w+v}{K} \right) \cdot v
	\end{align*}

	Dabei beschreibt der Faktor \(r_v\) die Reproduktionsrate der Beute (positives Vorzeichen) und der Faktor \(r_w\) die Sterberate der Räuber (negatives Vorzeichen), denn diese verhungern falls sie kein Futter finden.

	Nun stehen diese zwei Arten miteinander in Wechselwirkung. Dabei beschreibt der Faktor \(\alpha\) die Wahrscheinlichkeit einer Begegnung. Dieser ist einzig von der Gestalt des Biotops anhängig und für beide Arten gleich. Zusätzlich beschreibt die Konstante \(l_{v}\) wie viele Beutetiere bei der Begegnung gefressen werden (negatives Vorzeichen) und die Konstante \(l_{w}\) wie stark sich die Räuber nach der Mahlzeit vermehren (positives Vorzeichen). Mit diesen zusätzlichen Variablen ergeben sich folgende Gleichungen:
	 \begin{align*}
		\dot{w} &= -r_w \cdot \left(1 - \frac{w+v}{K} \right) \cdot w + l_w \cdot \alpha \cdot w \cdot v \\
		\dot{v} &= +r_v \cdot \left(1 - \frac{w+v}{K} \right) \cdot v - l_v \cdot \alpha \cdot w \cdot v
	\end{align*}

	Werden diese zwei Gleichungen mit der allgemein bekannten Lotka-Volterra Gleichung \cite{PredatorPrey} Seite 3 verglichen, ist leicht zu erkennen, dass diese damit identisch sind.

	\newpage

	Wie man in Abbildung \ref{fig:beute_raeuber_oszillation} erkennen kann, ist mit entsprechenden Parametern ein Schwingungsverhalten zu beobachten. Wobei die Phasen der Schwingungen der beiden Arten jeweils verschoben sind.
	
	Parameter benennen...

	\begin{figure}[!h]
  		\centering
 		\includegraphics[width=5.5cm]{Diagramme/beute_raeuber_oszillation.png}
  		\caption{Population von Räuber und Beute als Funktion der Zeit.}
  		\label{fig:beute_raeuber_oszillation}
	\end{figure}
	
	\newpage
	
	\section{\uppercase{Agentenbasierte Modellierung}}
	
	\newpage
	In der agentenbasierten Modellierung werden Gleichungen im Gegensatz zur analytischen Modellierung mit einzelnen Individuen gelöst.
	Damit können zusätzlich reel auftretende Effekte wie Überalterung oder ein Mindestalter zur Fortpflanzung berücksichtigt werden.\\
	Wir haben unser agentenbasiertes Modell speziell auf den Beute-Räuber Effekt angewendet, da dieser Effekt analytisch nur schwer zu
	lösen ist und Effekte wie Überalterung dort eine wichtige Rolle spielen. Die Individuen erhalten durch ihr Alter spezielle Eigenschaften:
	\begin{itemize}
		\item Hat das Individuum das maximale Alter überschritten, stirbt es. Es isst also nicht mehr und ist nicht in der Lage, sich Fortzupflanzen.
		\item Ist das Individuum unter einem gewissen Alter, wird es bei der Fortpflanzung nicht berücksichtigt, es frisst aber die normale Menge.
		\item Jedes Individuum zwischen dem Mindestalter zur Fortpflanzung und dem maximalen Alter frisst (als Räuber) andere Tiere und ist in der Lage, sich fortzupflanzen.
	\end{itemize}
	Die Menge der gefressenen Tiere $k$ wird wie wie bei den allgemeinen Formeln zum Beute-Räuber Modell bestimmt:
	$$k = l_v\cdot \alpha \cdot w_{lebend} \cdot v_{lebend}$$
	Bei der Fortpflanzung gelten auch die selben Formeln wie beim Beute-Räuber Modell, allerdings werden dabei nur die erwachsenen Individuen betrachtet. Außerdem wird das Verhungern von Tieren durch Einführung einer Menge $\beta$, die jedes Tier in einer Zeiteinheit zum Überleben fressen muss, berücksichtigt:
	$$\dot{w} = -r_w \cdot \left(1-\frac{w_{lebend} + v_{lebend}}{K}\right)\cdot w_{erwachsen} + l_v\cdot \alpha \cdot v_{lebend}\cdot w_{lebend} - \beta \cdot w_{lebend}$$
	$$\dot{v} = r_v \cdot \left(1-\frac{w_{lebend} + v_{lebend}}{K}\right)\cdot v_{erwachsen} - k$$

	\subsection{Implementation}
	Die agentenbasierte Simulation wurde mit Python 3.6 umgesetzt. Die einzelnen Einheiten wurden dabei durch einen einzigen Gleitkommawert, der ihr Alter angibt, in einem sortierten Array für jede Tierart repräsentiert. Alle Konstanten können in einem json-File angepasst werden und es können auch weitere Tierarten ohne zusätzlichen Programmieraufwand hinzugefügt werden. Die Ergebnisse der Simulation wurden als CSV-Datei abgespeichert und in Matlab visualisiert.\\
	In einer ersten Version waren alle Tiere noch in einem unsortierten Array und jedes Tier war Instanz einer extra entworfenen Klasse. Diese Methode war aber relativ langsam und  Berechnungen mit bis zu 100.000 Tieren über 3000 Zeitschritte dauerten etwa eine Stunde auf meinem Laptop. Mit der Optimierung über ein sortiertes Array und Darstellung der Tiere als Gleitkommazahl konnte ich die Rechenzeit für die selben Werte auf 52s reduzieren. \\
	Die verwendeten Rechenregeln sind nicht streng deterministisch. Die Auswahl der Tiere, die gefressen werden, erfolgt rein zufällig. Darum kann er beim erneuten Ausführen der Simulation zu minimal unterschiedlichen Ergebnissen führen. Link zu der Simulation finden Sie in unserem Github.
	\keywords{bla bla}
	\subsection{Ergebnis}
	In dem im Diagramm dargestellten Beispiel sieht man eine ungebremste Schwingung. Diese tritt nur in sehr kleinen Wertebereichen auf. Meistens ist die Schwingung gedämpft und pendelt sich auf einen festen Wert ein. Im dargestellten Ergebnis wurden 45\% der Beute aufgefressen und 55\% starben am Alter.\\
	Auffallend bei den Diagrammen ist, dass das maximale Alter der Räuber immer näherungsweise ein ganzzahliges Vielfaches der Periodendauer ist. Dies macht auch Sinn, da bei einem großen Anstieg der Bevölkerung viele Einheiten mit dem selben Alter entstehen. Diese Einheiten sterben also auch wieder gemeinsam, was eine Schwingung anregt. Da aber schon während des Sterbens mehr neue Einheiten erschaffen werden (und auch danach noch), ist diese Schwingung alleine betrachtet gedämpft. \\
	Eine weitere besondere Eigenschaft ist, dass bei Anfangswerten, die nicht zu weit vom Gleichgewicht entfernt sind, und ohne Aussterben der Beute nie Jäger verhungert sind. 
	\subsection{Ausblick}
	Bis zur Präsentation möchte ich noch mehrere Lebensräume mit unterschiedlichen Eigenschaften, zwischen denen die Tiere wandern können, einbauen. Der Aufbau der Simulation lässt dies zu, allerdings fehlte mir die Zeit, die Regeln für das wandern zwischen Lebensräumen zu definieren und zu implementieren. Dafür wollen wir auch paralleles Rechnen in mehreren Threads verwenden, was unter anderem ein Grund ist, warum wir Python für diese Simulation verwendet haben.\\
	
	\newpage
	
	\section{\uppercase{Der zelluläre Automat}}
	
	Weiter wird untersucht, ob ein Schwingungsverhalten wie in Abschnitt \ref{sec:Raeuber_Beute} beschrieben, auch bei einem zellulären Automaten zu beobachten ist. Dafür wird ein WATOR System verwendet. Die hier verwendete Implementation ist der Beschreibung aus \cite{Wator} nachempfunden hat aber durchaus auch Ähnlichkeiten mit der Implementation aus \cite{PlanetWator}.
	
	\subsection{Implementation}
	Nachfolgend wird auf die Einzelheiten der Implementation des zellulären Automaten eingegangen, dabei wird die Spezies der Räuber als Haie bezeichnet und die der Beutetiere als Fische.
	
	Als Datenstruktur des Automaten dient eine \(m \times m\) Matrix, wobei \(m\) die Größe des Lebensraums definiert. Diese \(m \times m\) Matrix hat wiederum drei Ebenen.
	
	Es existiert an der Stelle \(i,j\) der Matrix ein ...
	\begin{enumerate}[label=...]
		\item Fisch, falls der Wert \((i,j,1) > 0\) ist. Dieser Wert wird als Brutzeit der Fische bezeichnet.
		\item Hai, falls der Wert \((i,j,2) > 0\) ist. Dieser Wert wird als Brutzeit der Haie bezeichnet.
	\end{enumerate}
	Wobei an der Stelle \(i,j\) niemals ein Fisch und ein Hai gleichzeitig sein kann.

	Der Wert an der Stelle \((i,j,3)\) wird als zusätzliche Information für den Hai an der Stelle \(i,j\) benötigt und wird als Fastenzeit bezeichnet.

	Als angrenzende Felder der Stelle \(i,j\) sind die 4 Felder \((i+1,j), (i-1,j), (i,j+1), (i,j-1)\) definiert, wobei \( \forall i,j \in [1,m] \) gilt und bei Überlauf beziehungsweise Unterlauf wieder bei \(1\) beziehungsweise \(m\) weiter gezählt wird.
	
	Zusätzlich werden folgende Konstanten für die Simulation benötigt: \newline
	Reproduktionszeit der Fische: Sobald der Brutzeit-Wert eines Fisches unter diesen Wert fällt, wird ein neuer Fisch geboren. \newline
	Reproduktionszeit der Haie: Sobald der Brutzeit-Wert eines Haies unter diesen Wert fällt, wird ein neuer Hai geboren. \newline
	Überlebenszeit ohne Essen: Sobald der Fastenzeit-Wert eines Haies unter diesen Wert fällt, stirbt der Hai.
	
	Die Fische sowie die Haie werden jeweils in einem separaten Durchlauf berechnet. Es werden also zuerst alle Fische, danach alle Haie berechnet. Dabei ist die Rechenreihenfolge in jedem Durchlauf zufällig. Um Kollisionen zu vermeiden, werden jeweils alle neu berechneten Werte in eine neue Matrix geschrieben. Um nun leere angrenzende Felder zu finden, wird jeweils immer die alte sowie die neue Matrix betrachtet. So wird vermieden, dass sich ein Fisch oder Hai auf das gleiche Feld bewegen kann. Dieses Implementationsdetail unterscheidet sich von \cite{PlanetWator} wo alleine mit Wahrscheinlichkeiten gerechnet wird.
	
	
	\subsubsection{Das Verhalten der Fische}
	Jeder Fisch durchläuft folgendes durch Pseudocode beschriebenes Verhalten:

	\begin{small}
	\begin{verbatim}
		Dezimiere den Brutzeit-Wert des Fisches.
 		
		Falls ein leeres angrenzendes Feld existiert:
		    Bewege auf zufälliges leeres Feld.
 			
		    Falls der Brutzeit-Wert kleiner ist
		    als die Reproduktionszeit für Fische:
		        Ein neuer Fisch wird auf einem zufälligen
		        leeren angrenzenden Feld geboren.
		        Der Brutzeit-Wert wird auf einen
		        zufälligen Startwert zurückgesetzt.
 			
		sonst:
		    Bleibe stehen. 		
 		
	\end{verbatim}
	\end{small}
	
	\subsubsection{Das Verhalten der Haie}
	Jeder Hai durchläuft folgendes durch Pseudocode beschriebenes Verhalten:

	\begin{small}
	\begin{verbatim}
		Dezimiere den Brutzeit-Wert des Haies.
		
		Falls auf einem angrenzenden Feld ein Fisch existiert:
		    Gehe auf das Feld mit dem Fisch.
		    Der Fisch auf diesem Feld stirbt.
		    Der Fastenzeit-Wert wird auf einen
		    zufälligen Startwert zurückgesetzt.
		    
		Falls ein leeres angrenzendes Feld existiert:
		    Dezimiere den Fastenzeit-Wert.
		    Falls der Fastenzeit-Wert kleiner ist als 
		    die Überlebenszeit ohne Essen:
		        So stirbt der Hai.
		        
		    sonst:
		        Gehe auf zufälliges leeres Feld.
		        
		        Falls der Brutzeit-Wert kleiner ist
		        als die Reproduktionszeit für Haie:
		            Ein neuer Hai wird auf einem zufälligen
		            leeren angrenzenden Feld geboren.
		            Der Brutzeit-Wert wird auf einen
		            zufälligen Startwert zurückgesetzt.
		
		sonst:
		    Bleibe stehen. 
		
	\end{verbatim}
	\end{small}
	
	\newpage
	
	\subsection{Auswertung}
	Für die Simulation wurde eine \(128 \times 128\) Matrix verwendet. Als Ausgangslage wurden jeweils 128 Fische sowie 128 Haie an zufälliger Position mit zufälligem Brutzeit-Wert verwendet. In den ersten 500 Zyklen (siehe Abbildung \ref{fig:wator_diagram_100}) scheint das Verhalten noch recht zufällig:
	\begin{figure}[!h]
  		\centering
 		\includegraphics[width=5.5cm]{Diagramme/wator_diagram_100.png}
  		\caption{Die ersten 500 Zyklen der Wator-Simulation.}
  		\label{fig:wator_diagram_100}
	\end{figure}
	
	\begin{figure}[!h]
  		\centering
 		\includegraphics[width=5.5cm]{Diagramme/wator_image_100.png}
  		\caption{Die Wator-Simulation nach 500 Zyklen: Der Brutzeit-Wert der Fische ist als Graustufenskala von Weiss (Maximum) bis Grau (Minimum) dargestellt. Der Fastenzeit-Wert der Haie ist als Graustufenskala von Schwarz (Maximum) bis Grau (Minimum) dargestellt. }
  		\label{fig:wator_image_100}
	\end{figure}
	
	Wie in Abbildung \ref{fig:wator_image_100} zu erkennen ist, sind die Fisch sowie die Hai Ansammlungen der Simulation jedoch nicht mehr völlig zufällig. Es bilden sich immer mehr Fisch- und Hai-Fronten, welche sich als Gruppe fortbewegen. Tatsächlich bildet sich auch nach weiteren 1000 Zyklen ein deutliches Schwingungsverhalten ab, wie die Abbildung \ref{fig:wator_diagram_348} zeigt.
	\begin{figure}[!h]
  		\centering
 		\includegraphics[width=5.5cm]{Diagramme/wator_diagram_348.png}
  		\caption{Schwingungsverhalten nach weiteren 1000 Zyklen.}
  		\label{fig:wator_diagram_348}
	\end{figure}

	\newpage

	\subsection{Schlussfolgerungen}
	bla bla bla

	\section{\uppercase{Fazit}}

	\vfill
	\bibliographystyle{apalike}
	{\small
	\bibliography{literatur}}

\end{document}	

