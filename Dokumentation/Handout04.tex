\documentclass[a4paper]{article}

\usepackage[utf8]{inputenc}
\usepackage[ngerman]{babel}

\usepackage{epsfig}
\usepackage{subfigure}
\usepackage{calc}
\usepackage{amssymb}
\usepackage{amstext}
\usepackage{amsmath}
\usepackage{amsthm}
\usepackage{multicol}
\usepackage{pslatex}
\usepackage{apalike}
\usepackage{enumitem}
\usepackage{MOSI_handout}     % Please add other packages that you may need BEFORE the MOSI.sty package.

\newcommand{\team}{Jannik Simon Münz, Ruedi Lüthi}
\newcommand{\theme}{Ausbeutung von Tierpopulationen}
\setcounter{page}{1}

\subfigtopskip=0pt
\subfigcapskip=0pt
\subfigbottomskip=0pt

\begin{document}

	\title{Ausbeutung von Tierpopulationen}
	
	\author{\authorname{Jannik Simon Münz und Ruedi Lüthi}}
	
	\keywords{Wachstum, Räuber und Beute, agentenbasierte Modellierung, Zellulärer Automat, WATOR}
	\abstract{In dem nachfolgenden Bericht wird die konkrete Implementation einer agentenbasierten Modellierung sowie die eines zellulären Automaten beschrieben. Dabei werden die Resultate der beiden Modelle mit dem klassischen Räuber-Beute Modell verglichen. Dabei konnte beim Betrachten aller Modelle jeweils immer ein Schwingungsverhalten beobachtet werden.}

	\onecolumn \maketitle \normalsize \vfill

	\section{\uppercase{Wachstumsgleichung}}
	\noindent Definition der Konstanten:
	\begin{flalign*}
		y(t) &:= \textrm{Population zum Zeitpunkt }t & \\
		r &:= \textrm{Wachstumsfaktor} & \\
		K &:= \textrm{Umweltkapazität} &
	\end{flalign*}
	Differentialgleichung zum logistischen Wachstum:
	\begin{align*}
		y(t+\Delta t) &= y(t) + r\cdot\left(1-\frac{y(t)}{K}\right) \cdot y(t) \cdot \Delta t \\
		%&= \left(1 + r\cdot\left(1-\frac{y(t)}{K}\right) \cdot \Delta t \right) \cdot y(t)
		\frac{y(t+\Delta t) - y(t)}{\Delta t} &= r \cdot\left(1-\frac{y(t)}{K}\right) \cdot y(t) \\
		\dot{y} &= r \cdot\left(1-\frac{y}{K}\right) \cdot y \\
	\end{align*}

	\noindent Erweiterung durch die Fangintensität \(E\):
	\begin{align*}
		%y(t+\Delta t) &= y(t) + r \cdot \left(1-\frac{y(t)}{K}\right) \cdot y(t) \cdot \Delta t - E \cdot y(t) \cdot \Delta t \\
		%&= \left(1+r \cdot \left(1-\frac{y(t)}{K}\right)\cdot \Delta t - E \cdot \Delta t \right) \cdot y(t)
		\dot{y} &= r \cdot\left(1-\frac{y}{K}\right) \cdot y - \underbrace{E \cdot y}_{\textrm{Ertrag}}
	\end{align*}	
	
	\noindent Umstellen auf die Form der Anfangsgleichung:		
	\begin{align*}
		%&= \left(1+r \cdot \left(\underbrace{1 - \frac{E}{r}}_{\textrm{ausklammern}} - \frac{y(t)}{K} \right) \cdot \Delta t \right) \cdot y(t) \\		
		%&= \left(1+\underbrace{r \cdot \left( 1 - \frac{E}{r} \right)}_{\tilde{r}} \cdot \left(1 - \frac{y(t)}{\underbrace{K \cdot \left( 1 - \frac{E}{r} \right)}_{\tilde{K}}} \right) \cdot \Delta t \right) \cdot y(t) \\		
		\dot{y} &= \left( r \cdot\left(1 - \frac{y}{K}\right) - E\right) \cdot y \\
		&= r \cdot \left( 1 - \frac{E}{r} - \frac{y}{K} \right) \cdot y \\
		&= \underbrace{r \cdot \left( 1 - \frac{E}{r} \right)}_{\tilde{r}} \cdot \left( 1 - \frac{y}{\underbrace{K \cdot \left(1 - \frac{E}{r}\right)}_{\tilde{K}}} \right) \cdot y
	\end{align*}
	
	\noindent Stabiles Gleichgewicht bei: \(y = \tilde{K} = K \cdot \left(1 - \frac{E}{r}\right) \)
	
	\noindent Maximaler Ertrag \(= E \cdot \underbrace{K \cdot \left(1 - \frac{E}{r}\right)}_{=y}\) 
	
	\noindent Optimales \(E\) wird gefunden, durch das lösen der Maximalwertaufgabe der Funktion 
	\begin{align*}
		f(E) = E \cdot K \cdot \left(1 - E \cdot \frac{1}{r}\right)
	\end{align*}
	Ableiten und gleich null setzen:
	\begin{align*}
		&f'(E) = K - 2 \cdot E \cdot \frac{K}{r} = K \left(1 - E \cdot \frac{2}{r} \right) \stackrel{!}{=} 0 \\
		&\Rightarrow 1 - E \cdot \frac{2}{r} = 0 \Rightarrow E = \frac{r}{2}
	\end{align*}
	Maximaler Ertrag für \(E=\frac{r}{2}\):
	\begin{align*}
		\frac{r}{2} \cdot K \cdot \left(1 - \frac{r}{2}\cdot\frac{1}{r}\right) = \frac{r \cdot K}{4}
	\end{align*}
	
	\section{\uppercase{Räuber und Beute}}\label{sec:Raeuber_Beute}
	\noindent Definition der Konstanten:
	\begin{flalign*}
		w &:= \textrm{Population der Räuber (zum Zeitpunkt }t\textrm{)} & \\
		v &:= \textrm{Population der Beute (zum Zeitpunkt }t\textrm{)} & \\
		r_w &:= \textrm{Sterberate der Räuber} & \\
		l_w &:= \textrm{Reproduktionsrate der Räuber nach einem Mahl} & \\
		r_v &:= \textrm{Reproduktionsrate der Beute} & \\
		l_v &:= \textrm{Sterberate der Beute durch gefressen werden} & \\
		\alpha &:= \textrm{Wahrscheinlichkeit einer Begegnung}
	\end{flalign*}
	 \begin{align*}
		\dot{w} &= -r_w \cdot \left(1 - \frac{w+v}{K} \right) \cdot w + l_w \cdot \alpha \cdot w \cdot v \\
		\dot{v} &= +r_v \cdot \left(1 - \frac{w+v}{K} \right) \cdot v - l_v \cdot \alpha \cdot w \cdot v \\
	\end{align*}

	\noindent Gleichgewicht der Räuber wenn \(\dot{w} \stackrel{!}{=} 0\) gilt:
	\begin{align*}
		-r_w \cdot \left(1 - \frac{w+v}{K} \right) \cdot w + l_w \cdot \alpha \cdot w \cdot v &\stackrel{!}{=} 0\\
		-r_w \cdot \left(1 - \frac{w+v}{K} \right) + l_w \cdot \alpha \cdot v &= 0\\
		\frac{l_w \cdot \alpha \cdot v}{r_w} &= 1 - \frac{w}{K}-\frac{v}{K}\\
		\frac{K \cdot r_w - l_w \cdot \alpha \cdot w \cdot v - v \cdot r_w}{r_w} &= w
	\end{align*}
	
	\noindent Sei \(v=0\), so ist \(w\) im Gleichgewichtspunkt:
	\begin{align*}
		w = \frac{K \cdot r_w - l_w \cdot \alpha \cdot w \cdot 0 - 0 \cdot r_w}{r_w} = \frac{K \cdot r_w}{rw} = K
	\end{align*}	
	
	
	
	\newpage
	
	\section{\uppercase{Agentenbasierte Modellierung}}
		
	\begin{small}
	\begin{verbatim}
Für jeden Zeitschritt:
  Erhöhe die aktuelle Zeit um delta_t

  Für jede Tierart:
    Für alle Tiere bei denen die Differenz der
    aktuellen Zeit zum Geburtszeitpunkt größer
    als das maximale Alter der Tierart ist:
      Das Tier stirbt.

    w = Anzahl der Tiere der Tierart.
    w_erw = Anzahl Tiere bei denen die Differenz der
    aktuellen Zeit zum Geburtszeitpunkt größer
    als das Erwachsenenalter der Tierart ist.

    Ist die Tierart ein Räuber:
      v = Anzahl der Beutetiere.
      Anzahl gefressener Beutetiere = w*v*alpha*lv
      Für jedes Beutetier welches gefressen wird:
        Ein zufälliges Beutetier stirbt.
            
      delta_pop = lw*alpha*w_erw*v - beta*w.

    Ist die Tierart ein Beutetier:
      delta_pop = rv*(1-(Alle Tiere/K))*w_erw.

    Ist delta_pop positiv:
      Für jedes neugeborene Tier:
        Ein neues Tier wird geboren und 
        die aktuelle Zeit als Geburtstermin 
        wird gespeichert.

    Ist die Veränderung in der
    Population negativ:
      Für jedes sterbende Tier:
        Ein zufälliges Tier stirbt.	
 		
	\end{verbatim}
	\end{small}
	
	\section*{Quellcode}
	\noindent GitHub: https://git.io/vNc4o	
	
	\newpage
	
	\section{\uppercase{Zelluläre Automat}}
	\subsection{Fische}
	\begin{small}
	\begin{verbatim}
Für jeden Zeitschritt:
  Dezimiere den Brutzeit-Wert des Fisches.
 		
  Falls ein leeres angrenzendes Feld existiert:
    Bewege auf zufälliges leeres Feld.
 			
    Falls der Brutzeit-Wert kleiner ist
    als die Reproduktionszeit für Fische:
      Ein neuer Fisch wird auf einem zufälligen
      leeren angrenzenden Feld geboren.
      Der Brutzeit-Wert wird auf einen neuen
      zufälligen Startwert zurückgesetzt.
 			
  sonst:
    Bleibe stehen. 		
 		
	\end{verbatim}
	\end{small}
	
	\subsection{Haie}
	\begin{small}
	\begin{verbatim}
Für jeden Zeitschritt:
  Dezimiere den Brutzeit-Wert des Haies.
		
  Falls auf einem angrenzenden Feld
  ein Fisch existiert:
    Gehe auf das Feld mit dem Fisch.
    Der Fisch auf diesem Feld stirbt.
    Der Fastenzeit-Wert wird auf einen neuen
    zufälligen Startwert zurückgesetzt.
		    
  Falls ein leeres angrenzendes Feld existiert:
    Dezimiere den Fastenzeit-Wert.
    Falls der Fastenzeit-Wert kleiner ist als 
    die Überlebenszeit ohne Essen:
      So stirbt der Hai.
		        
    sonst:
      Gehe auf zufälliges leeres Feld.
		        
      Falls der Brutzeit-Wert kleiner ist
      als die Reproduktionszeit für Haie:
        Ein neuer Hai wird auf einem zufälligen
        leeren angrenzenden Feld geboren.
        Der Brutzeit-Wert wird auf einen neuen
        zufälligen Startwert zurückgesetzt.
		
  sonst:
    Bleibe stehen. 
		
	\end{verbatim}
	\end{small}
	

\end{document}	

